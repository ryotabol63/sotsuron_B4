\chapter{緒言}


\indent
近年,様々な分野でICTの活用が進められ,医療分野における高度な医療の提供,製造業における生産性向上などに貢献している.
一方,保育の分野では,教育の高度化の観点でICTの導入が進んでおらず,保育園に預けた子供の様子を知る手立てが未だ保育士の主観による報告に限られているといった問題が発生している.そのため,子供の行動解析データを介した保育分野におけるICTの導入が求められている.行動解析のために用いるデータにはさまざまあるが,移動場所の傾向を解析することで,同様の移動をした子供のデータから交友関係を,移動場所の傾向から子供の嗜好・個性をつかめる.

以上より,本研究では,子供の移動場所の傾向から行動解析を行うことを目指す.
移動場所の傾向解析を行うには子供がどこにいるのか計測する必要がある.
子供の位置情報を高精度で計測することができれば,子供の移動場所を計測できるが,こちらは設備コストの問題から決定的な技術がまだ登場していない.\cite{ntt}他方,カメラによる記録を解析する手法もあるが,こちらは記録から解析可能なデータにするのが手作業で行われており,解析コストの問題が生じる.そこで,本稿では,設備コスト,消費電力の低い小型無線タグ(TWELITE)を利用して位置計測を自動化することを目的とする.\newline

本論文は以下に示す5章より構成される.

第1章は緒言であり,本研究の背景,目的及び概要について述べたものである.

第2章では本研究で使用したモデルとして,無線タグおよびMQTTプロトコルを用いた計測システムについてまとめた.

第3章では本研究で使用した計測機器であるTWELITEタグの計測精度および使用プロトコルであるMQTTプロトコルの通信の確実性やレイテンシをまとめた.
得られたデータの処理方法についてもここで述べた.

第4章では計測システムを用いて実験を行い,その評価を行った.

第5章は結言である.

以上,本論文の企図するところを概説した.
