\chapter{結言}
本稿では,子供の移動場所の傾向を解析して行動解析を行うという目標のもと,小型無線タグTWELITEを用いた位置情報推定に向け,機器の性能評価と転送プロトコルの提案,補間やフィルタリングを含めたシステムの構成と性能評価を行った.
まず,小型無線タグの性能評価においては,通信距離と電波強度の関係について,ある一定の距離(5m程度まで)は通信距離の増加に伴い電波強度が減少するがそれ以降は電波強度が横ばいになるという結果が得られた.
続いて,MQTTプロトコルについて,0.5秒以下の低レイテンシで確実性の高いプロトコルであることが確かめられ,本稿における転送プロトコルとしては十分な性能を示した.
小規模な屋内で実装したシステムの性能評価では,90.6\%という高い推定正解率を示し,子供の移動場所の傾向を示すためには十分な結果と考えられる.
推定不正解となった部分についても,距離比の差が小さいことに起因するため,今後大きな規模に拡大することにより改善することも考えられる.
本稿ではデータ処理について,状態空間モデルを用いたフィルタリングおよび線形補間を用いたが,フィルタや補間方法には今回検討したよりも多くの方法・ソースコードがあるため,その中から最適なものを検討していきたい.
また,今回は小規模な実験にとどまったが,今後さらに実験環境を広げた場合の実験も行っていきたい.
本研究の最終目標はプレーパーク,保育園等での子供の位置情報を推定し,その情報を集積・分析することである.
実際の測定環境はきれいな図形のような形をしているとは限らず,システムの性能評価を踏まえたカスタマイズが必要となるので,それも併せて取り組んでいきたい.
そのうえで,集積されたデータの解析を含めた解析をして評価可能なシステムとして提案したい.

